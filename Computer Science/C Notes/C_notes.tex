% Created 2019-08-02 Fri 15:48
% Intended LaTeX compiler: pdflatex
\documentclass[12pt]{book}
\usepackage[utf8]{inputenc}
\usepackage[T1]{fontenc}
\usepackage{graphicx}
\usepackage{grffile}
\usepackage{longtable}
\usepackage{wrapfig}
\usepackage{rotating}
\usepackage[normalem]{ulem}
\usepackage{amsmath}
\usepackage{textcomp}
\usepackage{amssymb}
\usepackage{capt-of}
\usepackage{hyperref}
\usepackage{minted}
\usepackage[margin=1in] {geometry}
\usepackage{parskip}
\linespread {1.5}
\setcounter{tocdepth} {6}
\setcounter{secnumdepth} {6}
\date{\today}
\title{}
\hypersetup{
 pdfauthor={},
 pdftitle={},
 pdfkeywords={},
 pdfsubject={},
 pdfcreator={Emacs 26.2 (Org mode 9.2.3)}, 
 pdflang={English}}
\begin{document}

\tableofcontents

\part{C Basics}
\label{sec:org0ab0391}
\subsubsection{\texttt{printf()} formatting}
\label{sec:orgb523f6e}
Use \texttt{\%} with symbols to print the variables in different format.
Example:
\begin{minted}[breaklines=true,breakanywhere=true]{c}
printf("%c", a)  //print a in format of character
\end{minted}
\end{document}
